% !TEX TS-program = context (luatex)
% !TEX encoding = UTF-8
\mainlanguage[es]

% macro todas small caps
\define[1]\allsc{{\sc\setcharactercasing[word]{#1}}}

% un salto creado para que no indente 
\usemodule[fancybreak]
\setupfancybreak[spacebefore=line]
\definefancybreak[salto][indentnext=no]

% separador
\define\separador{\salto{%
  \startalignment[middle]
    \dontleavehmode
    \externalfigure[sep] [lines=.2,grid=top]
  \stopalignment}}

% años
\definemargindata[years][inmargin][style={\tfx\sc}]

%ampersand
\define\amper{{\em \&}\space}


%% simbolo del arbol
\definesymbol[arbol]
  [\getnamedglyphdirect{phaistos}{u101F2}]

% Fuentes
\loadtypescriptfile[ebgaramond]

\definefontfeature
  [quality]
  [protrusion=quality]

\definefontfeature
  [deco]
  [swsh=yes]

\definefontfeature
  [eb-garamond-normal]
  [eb-garamond-normal,quality,deco]

\definefontfeature
  [eb-garamond-smallcaps]
  [eb-garamond-smallcaps,quality]

\setupbodyfont[ebgaramond,12pt]
\setupcapitals[sc=yes]
\setupbodyfontenvironment[default][em=italic]


%%%%%%%%%%%%%%%%
% layout
%%%%%%%%%%%%%%%%

% alineacion y fine tuning de justificados
\setupalign[lesshyphenation,hz,hanging]
\setupspacing[packed]

% numeros de pagina abajo al centro
\setuppagenumbering[location={footer,center}]

\setbreakpoints[compound] % preferir guiones como puntos de particion. 
\setuptolerance[verytolerant] 

% Tamaño de caja, margenes...
\setuplayout
	[backspace=30ex,
	width=68ex,
	topspace=12ex,
	height=140ex,
	header=1em,
	headerdistance=1em,
	footer=1em,
	footerdistance=2em,
	margindistance=1em,
	grid=yes,
	setups=correcto]

% parrafos
%\setupwhitespace[line]

%encabezados
\defineframedtext[cabeza][strut=no,frame=off,width={\hsize},offset=0pt]

\definehead[hdv]   [title] 
\setuphead[hdv]    [style={\tfd},
                    before={\begingroup\startcabeza},
                    after={\blank[-1em]{\tfc \fontchar{paragraph}} {\tfb {\em Diseñador Gráfico}}
                    \stopcabeza\blank[8*line]\endgroup}]
                    
\setuphead[title]   [page=no,continue=yes,after=,before={\blank[line]}]
\setuphead[subject] [page=no,continue=yes,style=\tf\sc,after=,before={\blank[line]}]

% finis
\define\finis{
    \mbox{}\vfill
    \startalignment[center]
    {\tfd \symbol[arbol]}\crlf
    {\tfxx  Última actualización: \date \crlf
    Diseñado en \from[ctx] usando tipos \from[ebg]}
    \blank[small]
    \from[chiq]
    \stopalignment
}

	
%% hyperlinks %%%%%%%%%%%%%%%%%%%%%%%%%%%%%%%%%%%%%%%%%%%%%%%%%%%%%%%%%%%%%%%%%%%%%%%%%%%%%%%%%%

\setupinteraction[state=start,color=darkcyan]

\useURL[olga]
    [mailto:olgamarinarango@gmail.com][]
    [Olga Marín Arango] 

\useURL[claudia]
    [mailto:cadenasilva@fibertel.com.ar][]
    [Claudia Cadena Silva] 
    
\useURL[lgd]
    [http://www.logangattis.com/][]
    [Logan Gattis Designs]

\useURL[chiq]
    [http://chiquitico.org/][]
    [Chiquitico.org]
    
\useURL[redada]
    [http://laredada.org/][]
    [LaREDada]
    
\useURL[aspu]
    [http://aspucol.org/][]
    [{\sc aspu}]
    
\useURL[lpaa]
    [http://agenciaenartes.com/][]
    [La Productora, Agencia en Artes]
    
\useURL[chi]
    [http://chiquitico.org][]
    [Editorial Chiquitico]
    
\useURL[ryl]
    [http://www.editoraryldiseno.com/][]
    [RyL Diseño]

\useURL[ctx]
    [http://en.wikipedia.org/wiki/ConTeXt][]
    [\CONTEXT]
    
\useURL[ebg]
    [http://www.georgduffner.at/ebgaramond/][]
    [{\sc EB Garamond}]
    
\useURL[me]
    [maito:elandi@chiquitico.org][]
    [elandi@chiquitico.org]
    
%%%%%%%%%%%%%%%%%%%%%%%%%%%%%%%%%%%%%%%%%%%%%%%%%%%%%%%%%%%%%%%%%%%%%%%%%%%%%%%%%%%%%%%%%%%%%%%%
%%%%%%%%%%%%%%%%%%%%%%%%%%%%%%%%%%%%%%%%%%%%%%%%%%%%%%%%%%%%%%%%%%%%%%%%%%%%%%%%%%%%%%%%%%%%%%%%
%%%%%%%%%%%%%%%%%%%%%%%%%%%%%%%%%%%%%%%%%%%%%%%%%%%%%%%%%%%%%%%%%%%%%%%%%%%%%%%%%%%%%%%%%%%%%%%%

\starttext

\starthdv[title={Andrés Conrado Montoya Acosta}]

\starttitle[title={Posición actual}]

{\em Director creativo y ejecutivo}, \from [chi].

\stoptitle

\starttitle[title={Última posición}]

{\em Director creativo} para {\em \from[lgd]} \amper {\em \from[lpaa]}. 

\stoptitle

\starttitle[title={Áreas de experiencia}]

Diseño de libros · Tipografía · Caligrafía · Diseño Web

\startsubject[title={Software propietario}]

{\sc Adobe:} InDesign, Illustrator, Photoshop, Acrobat.

\stopsubject 

\startsubject[title={Software Libre \amper\ otras habilidades técnicas}]

Gimp, Inkscape, Dia, ImageMagick, PdfTk. Lenguajes de composición tipográfica basados en \TEX, tales como  
\LATEX\space\amper\CONTEXT. Lenguajes de marcado y de programación: \allsc{html5}, \allsc{css3}, \allsc{sass}, Markdown. Fundamentos de: Java Script, Phyton, Ruby. {\em Power User} avanzado, capaz de aprender a manejar prácticamente cualquier pieza de software en muy poco tiempo, sin importar el sistema operativo).

\stopsubject

\stoptitle



\starttitle[title={Educación}]

\years{2007} Buenos Aires, Argentina: Talleres y lecciones de caligrafía con la maestra Silvia Cordero Vega y el maestro Ricardo Rousselout. Cursos de fabricación de libros en la Papelera Palermo.

\years{2006} Graduado como Diseñador Gráfico en la Universidad Nacional de Colombia, 
con énfasis en caligrafía, tipografía y diseño de libros.

\years{1997} Estudios no completados de Artes Plásticas, en la Universidad Nacional de Colombia. 

\stoptitle




\starttitle[title={Experiencia}]

\years{2008-presente} Propietario de la \from[chi] en Bogotá, Colombia.

\years{2012-presente} Creative Director para \from[lgd].

\years{2011-2012} Graphic designer para \from[lgd].

\years{2011-2013} Webmaster para la Asociación Cultural \from[redada], y la Asociación Sindical de Profesores Universitarios, \from[aspu].

\years{2010-2013} Diseñador gráfico \amper webmaster para \from[lpaa].

\years{2008-2010} Diseño de libros para el Centro Editorial de la Facultad de Ciencias Humanas, Universidad Nacional de Colombia, sede Bogotá. 

\years{2004-2010} Diseñador para el magazín {\em Educación Superior}, publicada por el Ministerio de Educación Nacional, bajo la dirección de la editora \from[olga].

\years{2004-2007}Diseñador Gráfico para la editorial \from[ryl]. 

\stoptitle

\starttitle[title={Referencias}]

Para obtener referencias, escribir a: \from[me].

\stoptitle

\stophdv

\finis

\stoptext

